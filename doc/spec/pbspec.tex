\documentclass[11pt]{article}
\usepackage{listings}
\usepackage{hyperref}
\usepackage{fullpage}

\lstdefinelanguage{pbcc}{
morekeywords={transform, if, else, for, from, to, through, case, do, where, return, or,
              MatrixRegion1D, MatrixRegion2D, MatrixRegion3D, cell, region, row, col},
sensitive=true,
morecomment=[l]{\#},
morecomment=[l]{//},
morestring=[b]',
morestring=[b]",
}
\lstset{language=pbcc}
\lstset{numbers=left}
\lstset{tabsize=2}
\lstset{commentstyle=\textit}
\lstset{literate=
{REFTransform}{{\ref{sec:transform}}}{3}
{REFRule}     {{\ref{sec:rule}}}{3}
{REFBody}     {{\ref{sec:body}}}{3}
}


\begin{document}
\title{PetaBricks Language Specification}
\author{Jason Ansel}
\maketitle
\tableofcontents

\newpage
\section{Syntax}

\subsection{Minimal PetaBricks Program}

The following is an example of a (near) minimal PetaBricks program. It
copies the matrix A to the matrix B.

\begin{minipage}{\linewidth}
\begin{lstlisting}
transform Copy2D                                                 REFTransform
to B[w,h]                                                        REFTransform
from A[w,h]                                                      REFTransform
{
  to(B.cell(x,y) b)                                              REFRule
  from(A.cell(x,y) a)                                            REFRule  
  {                                          
    b=a;                                                         REFBody
  }                                         
}
\end{lstlisting}
\end{minipage}
This defines the transform Copy2D, which can be invoked as follows:

\begin{minipage}{\linewidth}
\begin{lstlisting}
MatrixRegion2D a,b;
...
Copy2D(a,b);
\end{lstlisting}
\end{minipage}

This invocation will result in the rule defined on lines~5-9, to be invoked
on every pair of elements in $a$ and $b$.


\subsection{Transform Header}
\label{sec:transform}

\subsubsection{TRANSFORM Clause}

\begin{tabular}{| l | l |}
\hline
\bf Required & Yes                    \\
\bf Available In Version & 1.0        \\
\bf Syntax & \tt transform IDENTIFIER \\
\bf Example & \tt transform Foo       \\
\hline
\end{tabular}

~

\noindent
Specifies the externally visible name of the transform.

\subsubsection{FROM Clause}

\begin{tabular}{| l | l |}
\hline
\bf Required & Yes                    \\
\bf Available In Version & 1.0        \\
\bf Syntax & \tt from MATRIXDEFLIST     \\
\bf Example & \tt from A[w,h], B[w]     \\
\hline
\end{tabular}

~

\noindent
\noindent Defines the inputs of the transform.  The sizes of the outputs
are given as expressions containing unbound variables.  The compiler and
runtime must find a valid assignment for these unbound variables to match
the sizes of the inputs given.  The values assigned to these variables are
visible both in rule headers and in rule bodies.


\subsubsection{TO Clause}

\begin{tabular}{| l | l |}
\hline
\bf Required & Yes                    \\
\bf Available In Version & 1.0        \\
\bf Syntax & \tt to MATRIXDEFLIST     \\
\bf Example & \tt to C[w,h], D[w]     \\
\hline
\end{tabular}

~

\noindent Defines the outputs of the transform.  The sizes of the outputs
are given as expressions containing unbound variables that are set based on the size
of the inputs.

\subsubsection{THROUGH Clause}

\begin{tabular}{| l | l |}
\hline
\bf Required & No                          \\
\bf Available In Version & 1.0             \\
\bf Syntax & \tt through MATRIXDEFLIST     \\
\bf Example & \tt through E[w,h], F[w]     \\
\hline
\end{tabular}

~

\noindent Defines intermediate data used during the transform execution. This
data is temporary and neither an input nor an output.  The sizes are given
as expressions containing unbound variables that are set based on the size
of the inputs.

\subsubsection{GENERATOR Clause}

\begin{tabular}{| l | l |}
\hline
\bf Required & No \\
\bf Available In Version & 2.0            \\
\bf Syntax & \tt generator IDENTIFIERLIST \\
\bf Example & \tt generator Bar\\
\hline
\end{tabular}

~

\noindent Specifies the name of another transform (or list of transforms)
that should be used to generate inputs to this transform during training.
The outputs of the other transforms are paired left to right with the inputs
of this transform.  The other transforms inputs are seeded with random data
of the desired input size.

If this clause is not used, transforms will be given random data during
training.

\subsection{Rule Header}
\label{sec:rule}

\subsubsection{TO Clause}
\begin{tabular}{| l | l |}
\hline
\bf Required & Yes                          \\
\bf Available In Version & 1.0             \\
\bf Syntax & \tt to( MATRIXARGLIST ) \\
\bf Example & \tt to( A.cell(x,y) a, B.region(0, 0, x, y) b)\\
\hline
\end{tabular}

~

\noindent Specifies the outputs for the rule.

\subsubsection{FROM Clause}
\begin{tabular}{| l | l |}
\hline
\bf Required & Yes                          \\
\bf Available In Version & 1.0             \\
\bf Syntax & \tt from( MATRIXARGLIST ) \\
\bf Example & \tt from( A.cell(x,y) a, B.region(0, 0, x, y) b)\\
\hline
\end{tabular}

~

\noindent Specifies the inputs for the rule.

\subsubsection{RECURSIVE Clause}
\begin{tabular}{| l | l |}
\hline
\bf Required & No \\
\bf Available In Version & 1.0             \\
\bf Syntax & \tt recursive( EXPRESSION )   \\
\bf Example & \tt recursive( n )           \\
\hline
\end{tabular}

~

\noindent Provide a hint to to autotuner that the decision to use this
rule should be based on comparing the given EXPRESSION to some constant.
Note that as of version 2.0 this language feature should no longer be needed
except in special cases.


\subsubsection{WHERE Clause}
\begin{tabular}{| l | l |}
\hline
\bf Required & No                          \\
\bf Available In Version & 2.0             \\
\bf Syntax & \tt where EXPRESSION          \\
\bf Example & \tt where x<10               \\
\hline
\end{tabular}

~

\noindent Limits the range where this rule may be applied.

\noindent 

\subsection{Return-style Syntactic Sugar}

PetaBricks allows a shorter version of rule headers that eliminates
the need for a TO clause.  It is similar to function style return syntax.
 
\begin{minipage}{\linewidth}
\begin{lstlisting}
B.cell(x,y) from(A.cell(x,y) a) {
  return a;
}
\end{lstlisting}
\end{minipage}
Is desugared to:

\begin{minipage}{\linewidth}
\begin{lstlisting}
to(B.cell(x,y) _rv) from(A.cell(x,y) a) {
  _rv=a;
  return;
}
\end{lstlisting}
\end{minipage}

\subsection{OR-Keyword Syntactic Sugar}

PetaBricks provides an ``OR'' keyword to eliminate the need for redundant 
rule headers.  ``OR'' simply means repeat the last rule header. For example:
 
\begin{minipage}{\linewidth}
\begin{lstlisting}
to(B b) from(A a) {
  Foo(b,a);
} or {
  Bar(b,a);
}
\end{lstlisting}
\end{minipage}
Is desugared to:

\begin{minipage}{\linewidth}
\begin{lstlisting}
to(B b) from(A a) {
  Foo(b,a);
}
to(B b) from(A a) {
  Bar(b,a);
}
\end{lstlisting}
\end{minipage}

\subsection{Rule Body}
\label{sec:body}

\end{document}


